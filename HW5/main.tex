\documentclass[12pt]{article}
 \usepackage[margin=1in]{geometry} 
\usepackage{amsmath,amsthm,amssymb,amsfonts}
 
\newcommand{\N}{\mathbb{N}}
\newcommand{\Z}{\mathbb{Z}}

%%%%  SHORTCUT COMMANDS  %%%%
\newcommand{\ds}{\displaystyle}
\newcommand{\Zi}{\mathbb{Z}}
\newcommand{\arc}{\rightarrow}
\newcommand{\R}{\mathbb{R}}
\newcommand{\Ni}{\mathbb{N}}
\newcommand{\Q}{\mathbb{Q}}
\newcommand{\stirling}[2]{\genfrac{\{}{\}}{0pt}{}{#1}{#2}}
 
\newenvironment{problem}[2][Problem]{\begin{trivlist}
\item[\hskip \labelsep {\bfseries #1}\hskip \labelsep {\bfseries #2.}]}{\end{trivlist}}
 
\begin{document}
 
%\renewcommand{\qedsymbol}{\filledbox}
%Good resources for looking up how to do stuff:
%Binary operators: http://www.access2science.com/latex/Binary.html
%General help: http://en.wikibooks.org/wiki/LaTeX/Mathematics
 
\title{Homework 5}
\author{Tanner Kvarfordt - A02052217}
\maketitle
 
\begin{problem}{1}
Recall $\stirling{n}{k}$ denotes the number of partitions of an $n$-set into $k$ blocks.  
Please prove the following  identity (which is a recurrence relation for $\stirling{n}{k}$)
$$\stirling{n}{k} = \stirling{n-1}{k-1} + k\stirling{n-1}{k},$$
and be sure to indicate the range of $n$ and $k$ and also indicate any relationship between $n$ and $k$ that must hold for the identity to be true.
\end{problem}

\begin{proof}
For this proof, I'll adopt the metaphor of placing people into rooms, where $n$ denotes the number of people to be placed into $k$ rooms. The left hand side of the relation simply denotes the number of ways to partition $n$ people into $k$ rooms. For the right hand side of the relation, we'll condition\footnote{Hopefully I am using this term correctly. I simply mean we will add some condition into consideration.} on one of the people of being placed into a room independently from the others. Without loss of generality, we'll call this person George. It then follows that there are two scenarios for the placement of George into a room. The first is that he will have a room of all his own, which means that the rest of the $n-1$ people would be placed into $k-1$ rooms. This can be clearly mathematically represented as $\stirling{n-1}{k-1}$. The second scenario is that George will not be unique to his own room. This means that there are $n-1$ people to be placed into $k$ rooms, with the number of possibilities for this action given by $\stirling{n-1}{k}$. Coupled with the fact that there are $k$ possibilities for George to be placed into one of these rooms, we have $k\stirling{n-1}{k}$. It then follows that the sum total of possibilities is given by $\stirling{n-1}{k-1} + k\stirling{n-1}{k}$, as shown in the right hand side of the relationship. In the end however, it is clear that we are still just placing $n$ people into $k$ rooms, thereby proving the relationship that $\stirling{n}{k} = \stirling{n-1}{k-1} + k\stirling{n-1}{k}$. This holds for $n,k \in\mathbb{N}$.
\end{proof}

\newpage

\begin{problem}{2}
Suppose $r,s,n \in \N$.  Please prove the identity (1) in two ways: combinatorially, and algebraically.
\begin{eqnarray} \ds \sum_{i = 0}^n \binom{r}{i}\binom{s}{n-i} = \binom{r+s}{n} \label{Vandermonde} \end{eqnarray}.
\end{problem}

\begin{proof}[Proof 1]
To prove this identity combinatorially, I will adopt the metaphor of grocery shopping. Without loss of generality, think of $r$ as the number of your favorite, but expensive foods. Think of $i$ as the number of these foods that you can afford to purchase on some shopping trip. Think of $s$ as the number of generic, bland, but tolerable and affordable food items that you will choose from to fill the remaining space in your cart with. Think of $n$ as the total number of food items you need to purchase on some shopping trip. So you must choose $i$ food items out of $r$, and $n-i$ items out of $s$, and do so until you have $n$ items. This represents the left hand side of the identity, $\sum_{i = 0}^n \binom{r}{i}\binom{s}{n-i}$. Or, put another way, from a sum total of $r+s$ food items, you must choose $n$ of them, which is represented in the right hand side of the identity, $\binom{r+s}{n}$. It is clear then, that both sides of the identity are counting the same thing, and therefore the identity $\sum_{i = 0}^n \binom{r}{i}\binom{s}{n-i} = \binom{r+s}{n}$ is verified.
\end{proof}

\begin{proof}[Proof 2]
To prove this identity algebraically, multiply both sides of the identity by $x^n$:
\begin{eqnarray}
x^n\sum_{i = 0}^n \binom{r}{i}\binom{s}{n-i} = \binom{r+s}{n}x^n \nonumber
\end{eqnarray}
Then take the summation of both sides from $n=0$ to $n=r+s$:
\begin{eqnarray}
\sum_{n=0}^{r+s}\Bigg( x^n\sum_{i = 0}^n \binom{r}{i}\binom{s}{n-i} \Bigg) = \sum_{n=0}^{r+s}\binom{r+s}{n}x^n \nonumber
\end{eqnarray}
For now, disregard the left hand side of the current equation and focus on the right hand side. By the binomial theorem\footnote{See Dr. Brown's Meeting Notes Eleven}, it is clear that
\begin{eqnarray}
\sum_{n=0}^{r+s}\binom{r+s}{n}x^n = (1+x)^{r+s} \nonumber
\end{eqnarray}
Split the exponent:
\begin{eqnarray}
(1+x)^{r+s}=(1+x)^r (1+x)^s \nonumber
\end{eqnarray}
Then, again invoking the binomial theorem, we have
\begin{eqnarray}
(1+x)^r (1+x)^s = \Bigg(\sum_{j=0}^r \binom{r}{j}x^j \Bigg) \Bigg(\sum_{k=0}^s \binom{s}{k}x^k \Bigg) \nonumber
\end{eqnarray}
Now expand the summations:
\begin{multline}
\Bigg(\sum_{j=0}^r \binom{r}{j}x^j \Bigg) \Bigg(\sum_{k=0}^s \binom{s}{k}x^k \Bigg) \\ =  \Bigg( \binom{r}{0}x^0+\binom{r}{1}x^1+\binom{r}{2}x^2+\cdots+\binom{r}{r}x^r \Bigg) \Bigg( \binom{s}{0}x^0+\binom{s}{1}x^1+\binom{s}{2}x^2+\cdots+\binom{s}{s}x^s \Bigg) \nonumber
\end{multline}
Then multiply the two expansions together:
\begin{multline}
\Bigg( \binom{r}{0}x^0+\binom{r}{1}x^1+\binom{r}{2}x^2+\cdots+\binom{r}{r}x^r \Bigg) \Bigg( \binom{s}{0}x^0+\binom{s}{1}x^1+\binom{s}{2}x^2+\cdots+\binom{s}{s}x^s \Bigg) \\ = \binom{r}{0}\binom{s}{0}x^0 + \Bigg( \binom{r}{0}\binom{s}{1} + \binom{r}{1}\binom{s}{0}\Bigg)x^1 + \Bigg(\binom{r}{0}\binom{s}{2}+\binom{r}{1}\binom{s}{1}+\binom{r}{2}\binom{s}{0} \Bigg)x^2 + \cdots \nonumber
\end{multline}
From this pattern, we can conclude that the coefficient of $x^n$ would be
\begin{eqnarray}
\binom{r}{0}\binom{s}{n}+\binom{r}{1}\binom{s}{n-1}+\cdots+\binom{r}{n-1}\binom{s}{1}+\binom{n}{0}=\sum_{p=0}^n \binom{r}{p}\binom{s}{n-p} \nonumber
\end{eqnarray}
And from this we have the left hand side of the identity (since $p=i$), thereby proving the identity that
\begin{eqnarray} \ds \sum_{i = 0}^n \binom{r}{i}\binom{s}{n-i} = \binom{r+s}{n} \label{Vandermonde} \nonumber \end{eqnarray}
\end{proof}

\begin{problem}{3}
Suppose $n \in \N$. Please prove $\ds n2^{n-1} = \sum_{i \geq 0}\binom{n}{i}i$ combinatorially and algebraically.
\end{problem}

\begin{proof}[Proof 1]
To prove this equality combinatorially, I will adopt the metaphor of choosing people for a team, where that team has a single captain. Take $n$ to be the number of people available to potentially be on the team. Take $i$ to be the number of people who will be selected for the team. Condition\footnote{Boy, I've used this twice now. Sure would be awkward if I was using it wrong.} on whether the captain is chosen prior to selecting the rest of the team, or afterward.

To choose a team of $i$ people from a group of $n$ people and to choose one captain for that team, we have $\binom{n}{i}$, and $i$ possibilities for a captain, giving us $\binom{n}{i}i$ possibilities. We must also consider that depending on how many people, $i$, that we want on the team, we can have $i \leq n$, giving us a sum total of $\sum_{i \geq 0}\binom{n}{i}i$ possibilities for creating a team with one captain from $n$ people, which is the right hand side of the equality given in the problem statement, and which represents the scenario in which a team is picked first, and then a captain.

If we pick the captain for the team first, we have $n$ possibilities for which individual will be captain. This leaves us with $n-1$ people for whom there are each two possibilities: to be on the team, or to not be on the team, so $2^{n-1}$. Since we need a team and a captain, we therefore have $n2^{n-1}$ possibilities for choosing a team. This is the left hand side of the equality given in the problem statement.

It is clear then, that both sides of the equality are counting the same thing, and therefore the equality is valid.
\end{proof}

\begin{proof}[Proof 2]
To prove this equality algebraically, consider the following binomial theorem variant:
\begin{eqnarray}
(1+x)^n = \sum_{i \geq 0}\binom{n}{i}x^i \nonumber
\end{eqnarray}
Taking the derivative of both sides with respect to $x$, we have
\begin{eqnarray}
n(1+x)^{n-1} = \sum_{i\geq 0}\binom{n}{i}ix^{i-1} \nonumber
\end{eqnarray}
Now consider $x=1$
\begin{eqnarray}
n(1+(1))^{n-1} = \sum_{i \geq 0} \binom{n}{i}i(1)^{i-1} \nonumber
\end{eqnarray}
Simplified, we have
\begin{eqnarray}
n2^{n-1} = \sum_{i \geq 0} \binom{n}{i}i \nonumber
\end{eqnarray}
Which brings us to the equality given in the problem statement, therefore proving that it is valid.
\end{proof}

\end{document}