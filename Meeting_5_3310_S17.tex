\documentclass[10pt, AMS Euler]{article}
\textheight=9.25in \textwidth=7in \topmargin=-.75in
 \oddsidemargin=-0.25in
\evensidemargin=-0.25in
\usepackage{url}  % The bib file uses this
\usepackage{graphicx} %to import pictures
\usepackage{amsmath, amssymb}
\usepackage{theorem, concrete}


\setlength{\intextsep}{5mm} \setlength{\textfloatsep}{5mm}
\setlength{\floatsep}{5mm}


{\theorembodyfont{\rmfamily}
\newtheorem{definition}{Definition}[section]}
{\theorembodyfont{\rmfamily} \newtheorem{example}{Example}[section]}
{\theorembodyfont{\rmfamily} \newtheorem{lemma}{Lemma}[section]}
{\theorembodyfont{\rmfamily} \newtheorem{theorem}{Theorem}[section]}
{\theorembodyfont{\rmfamily} \newenvironment{proof}{\par{\it
Proof:}}{\nopagebreak[4]\rule{2mm}{2mm}}}
{\theorembodyfont{\rmfamily}
\newenvironment{solution}{\par{\bf{Solution:}}}{\nopagebreak[4]\rule{2mm}{2mm}}


%%%%  SHORTCUT COMMANDS  %%%%
\newcommand{\ds}{\displaystyle}
\newcommand{\Z}{\mathbb{Z}}
\newcommand{\arc}{\rightarrow}
\newcommand{\R}{\mathbb{R}}
\newcommand{\N}{\mathbb{N}}
\newcommand{\Q}{\mathbb{Q}}

%%%%  footnote style %%%%

\renewcommand{\thefootnote}{\fnsymbol{footnote}}

\pagestyle{empty}
\begin{document}


\noindent{\bf \large MATH 3310 -- Meeting Five}, 1/24/2017\\

\noindent {\bf The Point:} Logic encompasses an attempt to systematically, carefully, and accurately record the ideal for 
how we *should* think and how we come to know anything.  George Boole developed an algebra that reflected the 
relationships understood to hold in Logic.  With this algebra, in theory and in ideal situations, one can perform 
calculations to deduce the truth of arguments and of statements.\\



\noindent {\bf Flour and Eggs:}

\begin{description}

\item[Logical Operations.]  There are three logical operations ($\implies$, $\wedge$, and $\vee$) and one operator ($\neg$) 
as far as our intents and purposes are concerned.  The operations are \emph{mappings} (a.k.a., \emph{functions}) 
from $\mathcal{M} \times \mathcal{M}$ into $\mathcal{M}$, and the operator is a mapping from $\mathcal{M}$ into $\mathcal{M}$.
The set $\mathcal{M}$ is the set of all statements that are either true or false.  The truthinator $\Phi$ is a mystical function 
that maps $\mathcal{M}$ into $\{T,F\}$, where $T$ is notation for `true' and $F$ is notation for `false'.  
The behavior of the operations is described (`defined' might be more appropriate) below in their \emph{truth tables} (we should probably 
call them $\Phi$-tables).

\begin{center}
$\begin{array}{|c|c|c|} \hline P & Q & P \implies Q \\
\hline
T & T & T \\
F & T & T  \\
T & F & F   \\
F & F & T \\\hline
\end{array}$
\hspace{0.5in} 
$\begin{array}{|c|c|c|} \hline P & Q & P \wedge Q \\
\hline
T & T & T \\
F & T & F  \\
T & F & F   \\
F & F & F \\\hline
\end{array}$ 
\hspace{0.5in} 
$\begin{array}{|c|c|c|} \hline P & Q & P \vee Q \\
\hline
T & T & T \\
F & T & T  \\
T & F & T   \\
F & F & F \\\hline
\end{array}$
\end{center}

The operator $\neg$ works like this: $\neg : \mathcal{M} \to \mathcal{M}$, where, for $P \in \mathcal{M}$, $\neg P$ is the \emph{negation} of $P$.  
The negation of a statement is not exactly the logical opposite of the statement, but the negation of a statement should be true if the statement (non-negated) 
is false, and vice-versa.  In other terms $\Phi(P) = T \implies \Phi(\neg P) = F$ and $\Phi(P) = F \implies \Phi(\neg P) = T$.  
The negation of a statement can be tricky; simply putting a ``not'' in it somewhere may be accurate, but it may not be useful. 

\item[Logical System.]  The Logical System we officially, traditionally, work with in Mathematics consists of $\mathcal{M}$, $\Phi$, $\implies$, $\wedge$, $\vee$, and $\neg$.
We have a set of statements, a way to decide whether statements are true or false, and operations to build statements out of other statements.
We will see that some of the operations are not necessary in the sense that we could replace some with other combinations or operations or completely new operations.  
For example $\implies$ can be disregarded since $P \implies Q$ \emph{is (logically) equivalent to} $\neg P \vee Q$.  

\item[Logical Equivalence.]  \textsf{What is meant by `\emph{is logically equivalent to}'?}  A statement is {\bf logically equivalent} to another statement if 
the two statements, when analyzed with a truth table, yield the same $\Phi$-values.  For example ``$P \implies Q$'' and ``$\neg P \vee Q$'' are logically equivalent 
statements because they have the same truth tables:

\begin{center}
$\begin{array}{|c|c|c|c|} \hline P & Q & \neg P \vee Q & P \implies Q \\
\hline
T & T & T & T \\
T & F & F & F \\
F & T & T & T  \\
F & F & T & T \\\hline
\end{array}$
\end{center}

\textsf{Notation.}  For $P, Q \in \mathcal{M}$, the notation $P \equiv Q$ means $P$ and $Q$ have the same truth tables. 

\item[Logic is like Algebra on Statements.]  George Boole was one of the first mathematicians or logicians to attempt to produce a system like algebra that 
could be used on language in general to determine truth.  His work on this published in 1847 was arguably the first substantial move toward the digital computer. 
In the document \emph{A Brief Treatment of Logic}, section 4, I use the more common variables $x$, $y$, $z$, etc. to denote statements and `$=$' for `$\equiv$' to 
highlight the analogy.  So for example the calculation $x \implies y = \neg x \vee y$ is valid and so is $\neg \neg x = x$ because each statement of equivalence can be 
proved by examining truth tables.  For another example, what are often called ``\emph{DeMorgan's Laws}''\footnote{More loathsome terminology --- these are \emph{theorems}, 
not ``laws''.}:
$\neg (x \wedge y) = \neg x \vee \neg y$ and $\neg ( x \vee y ) = \neg x \wedge \neg y$.

\end{description}


\noindent {\bf Agenda items:}
\begin{enumerate}

\item {\bf Logical operations and Set Theoretic ones.}  Each logical operation we need corresponds to a set theoretic operation or relation.
Union and intersection are (set-theoretic) operations which take two sets and form a new one, but \emph{subset} indicates a relationship between two sets.
This is why I say ``operation or relation''.
In the context of Logic ``or'', ``and'', and If-Then are operations: each takes two statements and produces a new one.
The connection between Logic and Set Theory is of course a consequence of the definitions.

Let $A$ and $B$ be sets.  The {\bf union} of $A$ and $B$, denoted $A \cup B$, is the set $\{\mbox{objects $x$}: x \in A \mbox{ or } x \in B\}$.
The {\bf intersection} of $A$ and $B$, denoted $A \cap B$, is the set $\{\mbox{objects $x$}: x \in A \mbox{ and } x \in B\}$.

I concede that the fact $A \subseteq B$ is defined via an If-Then relationship ($x \in A \implies x \in B$) makes it difficult to see the connection.
I don't know what to say about this, other than \emph{think about how an If-Then statement is used.}  The calculus fact I used in \emph{Meeting Four} 
as an example names a containment relationship among objects; the objects are continuous functions and differentiable functions.  
Here is an abbreviation of the connections:

\begin{center}
\begin{tabular}{|c|c|l|}
  \hline
  % after \\: \hline or \cline{col1-col2} \cline{col3-col4} ...
  English & Notation & Set-Theoretic \\ \hline
  ``and'' & $\wedge$ & $\cap$ (intersection) \\
  ``or'' & $\vee$ & $\cup$ (union) \\
  If-Then & $\implies$ & $\subseteq$ (subset) \\ \hline

\end{tabular}
\end{center}

\item {\bf The Contrapositive.}  The {\bf contrapositive} of ``If $P$, then $Q$'' is ``If not $Q$, then not $P$''.  
In our logic-notation: the \emph{contrapositive} of  $P \implies Q$ is $\neg Q \implies \neg P$.

{\bf Theorem:} \emph{The statement $P \implies Q$ is logically equivalent to $\neg Q \implies \neg P$.} 

\emph{Proof:} Consider the truth table below and note that the third and fourth columns are identical.
The fourth column is derived by applying $\Phi$ to $\neg Q \implies \neg P$ after applying the appropriate 
$\Phi$-values dictated in applying $\neg$.  The boxed entries are the final $\Phi$-values.

\begin{center}
$\begin{array}{|c|c|c|c|} \hline P & Q & P \implies Q & \neg Q  \implies \neg P \\
\hline
T & T & T & F \; \boxed{T} \; F \\
T & F & F & T \; \boxed{F} \; F \\
F & T & T & T \; \boxed{T} \; T \\
F & F & T & T \; \boxed{T} \; T\\\hline
\end{array}$
\end{center} 

Therefore the statements $P \implies Q$ and $\neg Q \implies \neg P$ are equivalent. \hfill \rule{2mm}{2mm}

\textsf{So What:} That `$P \implies Q$' and `$\neg Q \implies \neg P$' are logically equivalent means, among other things, that whenever an 
If-Then statement is proved, its contrapositive can be used with impunity.  The statement \emph{If a function $f(x)$ is differentiable at $x = a$, then 
$f(x)$ is continuous at $x = a$} can be used to prove that $f(x)$ is continuous at $x = a$, but its contrapositive can be used to prove that a 
$f(x)$ is not differentiable at $x =a$. 


\item {\bf Axioms and the $\{M,I,U\}$-System.}  The $\{M,I,U\}$-system is intended to be a miniature Mathematics.  The set $\mathcal{S}$ is the
set of all MIU-theorems.  We will \emph{create} or \emph{discover} (depending on how you look at things) members of $\mathcal{S}$.
I'll prove an MIU-theorem using a format which, I think, enforces a very strict habit of mind.
It is an \textsc{unacceptable} format in ordinary circumstances because it is basically hieroglyphics.
I will train you how to read it.  The column headings are intended to be self-explanatory.  In the Justification-column will be the notation $(x;y, \textsc{blah})$,
where $x$ is the axiom that is applied, $y$ is the line to which it is applied, and \textsc{blah} is a detail intended to clarify how the axiom was applied.

\begin{center}
\begin{tabular}{|c|l|l|}
  \hline
  % after \\: \hline or \cline{col1-col2} \cline{col3-col4} ...
  Line & String & Justification \\ \hline
  1. & $MI$ & $(0,\mbox{not app.})$ \\
  2. & $MII$ & $(2;1,x = I)$ \\
  3. & $MIIII$ & $(2;2,x=II)$ \\
  4. & $MIIIIIIII$ & $(2;3,x=IIII)$\\
  5. & $MIIIIUI$ & $(3;4, x = MIIII, y = I)$\\
  6. & $MUIUI$ & $(3;5, x = M, y = IUI)$\\

\end{tabular}
\end{center}

Therefore $MUIUI \in \mathcal{S}$.  Of course, we also proved that $MII$, $MIIII$, $MIIIIIIII$, and $MIIIIUI$ are all in $\mathcal{S}$ as well.
Note that naming the line we apply an axiom to is always the line directly above.
This need not be the case, but there are no axioms that allow us to combine two strings or use two strings to produce a third.

Now, I'll write the derivation of $MUIUI$ in English which is the way we're supposed to write it. 

{\bf Theorem} \emph{$MUIUI \in \mathcal{S}$.}

\emph{Proof:}  By Axiom 0 $MI$ is a member of $\mathcal{S}$, and Axiom applied to this with $x = I$ yields $MII \in \mathcal{S}$.
A second and third application of Axiom 2, the first with $x = II$ and the second with $x = IIII$, yields $MIIIIIIII \in \mathcal{S}$.
Now we use Axiom 3 to collapse the substring of the fifth, sixth, and seventh $I$ into $U$ obtaining $MIIIIUI \in \mathcal{S}$.
Finally, we apply Axiom 3 to the substring of the first, second, and third $I$ to obtain the desired result.  \hfill \rule{2mm}{2mm}


\item{\bf Think Outside the Box.}  Can you produce some sort of test that you could use on a random string of $M$s, $I$s, and $U$s to determine whether
it is a member of $\mathcal{S}$?  Is $IIIUMIII \in \mathcal{S}$?  Is $MIUIUIU \in \mathcal{S}$?  With a little thought you will probably notice that
all MIU-theorems must begin with $M$.  This can be proved, but you cannot use the $MIU$-axioms, strictly, to prove it; you've got to step outside the
$\{M,I,U\}$-system to even state the fact ``\emph{All MIU-theorems must begin with $M$}''.  This is a \emph{metatheorem}.
\item {\bf The If-Then Operation.}   Below is what is often referred to as a `\emph{truth table}' for the most important logical operation, If-Then.
Suppose $P,Q \in \mathcal{M}$.  The table below is intended to display the $\Phi$-value of the statement $P \implies Q$ when the $\Phi$-values of
$P$ and $Q$ are known.

\textsf{Analogy:} The $\{M,I,U\}$-System is intended to model the axiomatic system of Mathematics.  
The set $\mathcal{S}$ is analogous to $\{P \in \mathcal{M}: \Phi(P) = T\}$ and since `$T$' means `true' which in turn means 
there exists a proof, the application of the axioms of the $\{M,I,U\}$-system is analogous to proving statements in our 
system of Mathematics using the axioms of our Mathematics together with Logic.  The size and scope of the $\{M,I,U\}$-System is 
conducive to suggesting what I think is a very powerful and deep idea.  

\item {\bf Metatheorems.}  Recall that, for our intents and purposes, and to bootstrap the idea, \emph{truth} is defined
as the existence of a proof.  A {\bf theorem} is a statement that has been proved.  So, if a statement has earned the distinction
of `theorem' it has been proved.  There's a limbo in which some statements exist; a statement in this limbo is a \emph{conjecture}.
A {\bf conjecture} is a statement, apparently mathematical, that has no proof and no counterexample.  Consider again \emph{Goldbach's Conjecture}:

\textsf{Not a Theorem ... yet:} \emph{Every even integer greater than $4$ can be expressed as the sum of two primes.}

This statement is clearly about mathematical objects and so if it is true, there \emph{should} be a proof, right?

Nope.  One of \emph{G\"{o}del's incompleteness theorems} tells us that Goldbach's Conjecture may be unprovable even if it is true.

{\bf Theorem:} (G\"{o}del's Theorem) \emph{In any axiomatic system that can support arithmetic, there exists statements that are true
but have no proof.}

To me, it is fascinating that, even though there is not a vast number of axioms that give us arithmetic, we cannot tell whether a
statement is one of the kind that G\"{o}del's Theorem promises.
The MIU-system, however, is not robust enough to support arithmetic and because of that we can produce \emph{metatheorems}, statements
about the MIU-system that can be used to determine whether a statement in MIU-language is in fact an MIU-theorem.

I don't have a definition for \emph{metatheorem}, well not really; the way I use `\emph{metatheorem}', however, is consistent with
how most use `\emph{meta}' in conjunction with another term --- assuming they are doing it correctly.  For example, `\emph{Metaphysics}'
for some is the study of the stuff in which experience happens --- the stuff that makes Physics (like Newtonian Physics, or Einsteinian Physics, or whatever)
possible in the first place.  Theories in \emph{metaphysics}, it could be argued, are \emph{about} Physics.
So, when \textsf{Prompt One} from \emph{An introduction to formal systems ...} asks you to prove a metatheorem, it is asking you to prove a theorem
\emph{about} the MIU-system; in particular, it asks for a theorem that about MIU-theorems.  Ideally, the metatheorem you prove will be a theorem that
\emph{could} help you determine whether a string of $M$s, $I$s, and $U$s is not an MIU-theorem.  The proof you produce will use the MIU-system's axioms, of course,
but, because it is a metatheorem, it will also use ideas and tools that are outside of the MIU-system's reach.  Logic, for example.  



\end{enumerate}

\noindent \underline{\hspace{3in}}\\

\noindent{\bf Homework \#2:} ({\bf Due 1/27/2017})  Reference the documents \emph{A Brief Treatment of Set Theory},
\emph{An introduction to formal systems ...}, and \emph{A Brief Treatment of Logic}.

\begin{enumerate}

\item Please respond to \textsf{Prompt One} from \emph{An introduction to formal systems ...}.

\item {\bf Exercise 6} from \emph{A Brief Treatment of Set Theory}.

\item {\bf Exercise 9} from \emph{A Brief Treatment of Set Theory}.

\item {\bf Exercise 6} from \emph{A Brief Treatment of Logic}.

%Exercises 3, 5, and 6 from \emph{A Brief Treatment of Logic}
%will be {\bf due Monday 1/19/2015}. \\

\end{enumerate}

\noindent \underline{\hspace{3in}}\\


%This lecture has been brought to you by ... Containers!

%\hspace{4in} \includegraphics[scale=0.4]{jars}

\end{document}