\documentclass[10pt, AMS Euler]{article}
\textheight=9.25in \textwidth=7in \topmargin=-.75in
 \oddsidemargin=-0.25in
\evensidemargin=-0.25in
\usepackage{url}  % The bib file uses this
\usepackage{graphicx} %to import pictures
\usepackage{amsmath, amssymb}
\usepackage{theorem, concrete, multicol}


\setlength{\intextsep}{5mm} \setlength{\textfloatsep}{5mm}
\setlength{\floatsep}{5mm}


{\theorembodyfont{\rmfamily}
\newtheorem{definition}{Definition}[section]}
{\theorembodyfont{\rmfamily} \newtheorem{example}{Example}[section]}
{\theorembodyfont{\rmfamily} \newtheorem{lemma}{Lemma}[section]}
{\theorembodyfont{\rmfamily} \newtheorem{theorem}{Theorem}[section]}
{\theorembodyfont{\rmfamily} \newenvironment{proof}{\par{\it
Proof:}}{\nopagebreak[4]\rule{2mm}{2mm}}}
{\theorembodyfont{\rmfamily}
\newenvironment{solution}{\par{\bf{Solution:}}}{\nopagebreak[4]\rule{2mm}{2mm}}


%%%%  SHORTCUT COMMANDS  %%%%
\newcommand{\ds}{\displaystyle}
\newcommand{\Z}{\mathbb{Z}}
\newcommand{\arc}{\rightarrow}
\newcommand{\R}{\mathbb{R}}
\newcommand{\N}{\mathbb{N}}
\newcommand{\Q}{\mathbb{Q}}
\newcommand{\stirling}[2]{\genfrac{\{}{\}}{0pt}{}{#1}{#2}}

%%%%  footnote style %%%%

\renewcommand{\thefootnote}{\fnsymbol{footnote}}

\pagestyle{empty}
\begin{document}


\noindent{\bf \large MATH 3310 -- Meeting Ten}, 2/9/2017\\

\noindent {\bf Topic:}  Distributions ... or ... Functions on Finite Sets.\\

%\noindent {\bf Key Ideas:}
%\begin{itemize}
%\item {\bf Definitions.}  Mathematics expands through good definitions.
%\end{itemize}


\noindent {\bf Flour and Eggs:}

\begin{description}
\item[Functions.]  Keep the concept of functions close to your mind's eye.

\item[Finite Sets.] I use the term `$n$-set' to mean a set with size $n$, or cardinality $n$, where $n \in \N$. 
An $n$-set we will work with often, 'cuz we typically lose no generality in it, is $[n] = \{1,2,3,\dots, n\}$ and 
we conventionally take $[0]$ to mean $\O$.
\end{description}

\noindent {\bf Agenda items:}
\begin{enumerate}
%\item{\bf Comment(s) Inspired by Homework Solutions.}

\item {\bf Distributions.}  I claimed that ``many, possibly all, counting problems can be seen as distribution problems'' not to 
correctly align your perspectives with those of the universe, but rather to provide another perspective.  
For example, let's think about choosing a subset of size $k$ from a set with $n$ elements (where $n$ and 
$k$ are nonnegative integers, of course).  Let $K$ denote the $k$-set and $N$ the $n$-set.  Clever, I know.  
The function $f:K \to N$ if regarded as one-to-one, chooses the subset: suppose $x \in K$, $y \in N$, and $(x,y) \in f$; 
the pair $(x,y)$ can be interpreted as ``element $y$ of $N$ belongs to the subset $K$''.  If $K$ is a set of balls, and $N$ is a set of 
boxes, then $(x,y)\in f$ means $f$ places ball $x$ into box $y$.  If a box of $N$ gets a ball under $f$, we can think of it as choosing that element of 
$N$ for the subset.  Note that $K$ should be regarded as a set of unlabeled objects and $N$ as a set of labeled objects. 

\item {\bf Binomial Coefficients.} The symbol $\binom{n}{k}$ is defined to be the number of $k$-element subsets
in an $n$-element set.
In many textbooks the symbol is defined to be $\binom{n}{k} = \frac{n!}{k!(n-k)!}$.
This is a lame\footnote{I'm using this term literally, not figuratively as it is sometimes used.
I mean that defining $\binom{n}{k}$ in this way makes the symbol basically useless, but the symbol has a meaning which is
hidden by the bandages covering $\binom{n}{k}$ by this definition -- now I'm speaking figuratively.} definition.

\textsf{Terminology:} I will call a (finite) set with $k$ elements a \emph{$k$-set.}

The symbol $\binom{n}{k}$ is called a \emph{binomial coefficient} because it is the coefficient of the term of degree $k$ in a binomial raised
to the power $n$.  For example, the coefficient on $x^4$ in $(1+x)^n$ is $\binom{n}{4}$.  You should convince yourself of this ASAP. 
Think about each coefficient $C_i$, $0 \leq i \leq n$, in the expansion
$$(1+x)^n = C_0 + C_1x +C_2 x^2 + C_3x^3 +C_4 x^4 + \cdots + C_ix^i + \cdots C_nx^n.$$
Realize that the coefficient $C_i$ is the number of ways to get a term of the form $x^i$ from the product
$$\underbrace{(1+x)(1+x)(1+x) \cdots (1+x)}_{\mbox{$n$ factors}},$$
and this is the number of ways to choose $x$ from $i$ of the factors.  Therefore $C_i = \binom{n}{i}$, and 
$$(1+x)^n = \sum_{i = 0}^n\binom{n}{i}x^i.$$

{\bf Pascal's Triangle.} The binomial coefficient $\binom{n}{k}$ satisfies the following relationship, which gives a way to determine any
binomial coefficient from other binomial coefficients with smaller upper and lower indexes.  In other terms, the theorem defines $\binom{n}{k}$
\emph{recursively}. 

{\bf Theorem 2.9.1.} (Pascal's Identity) \emph{For $n,k \in \N$ and $n \geq 1$, $\binom{n}{k} = \binom{n-1}{k-1}+ \binom{n-1}{k}$.}

Pascal's Identity yields a way to display the binomial coefficients in a triangular array, using the facts $\binom{n}{0}= \binom{n}{n} = 1$
and $\binom{n}{k} = 0$ if $k > n$.

\emph{Proof of Pascal's Identity}: Suppose you have $n-1$ objects and a llama --- so $n$ objects in total, one of which happens to be a llama.
How many $k$-sets can you create from these $n$ objects?  By definition, $\binom{n}{k}$.  But we could condition on the llama and whether or not
we want it in the $k$-set. If we don't, there are $\binom{n-1}{k}$ ways to create a $k$-set with no llama, since we
simply disregard the llama during the selection process.  If we want the llama, then we put the llama in the set and then
select the remaining $k-1$ objects from $n-1$ other non-llama objects.  In all we have $\binom{n-1}{k}+\binom{n-1}{k-1}$ to build a $k$-set
if we condition on the llama.  But in the end we still have a $k$-set selected from $n$ objects, and the identity is established.  Note that we needed
at least one object -- the llama -- to make this argument. \hfill \rule{2mm}{2mm}

\item{\bf Factorial: Linear Arrangements.} Pretend $n$ objects are arranged linearly, or ranked as first, second, third, and so on.
Define the symbol $n!$, read ``$n$-factorial'', to be the number of ways to rearrange the $n$ objects.

{\bf Theorem 2.7.1.} \emph{For $m \in \Z^+$, $n! = n(n-1)(n-2) \cdots 2 \cdot 1$, and $0! =1$.}

\emph{Proof}: Take the $n$ objects and place them in a set.  Next decide which of the $n$ objects is to be the new first object.  Having
done that decide which of the remaining $n-1$  objects is to be the new second object.  Continuing in this manner observe that after
we place the new $i$th object, we have $n-i$ choices for the new $(i+1)$st object, where $1 \leq i \leq n-1$. \hfill \rule{2mm}{2mm}

Define the symbol $n^{\underline{k}}$, read ``$n$ to the $k$-falling'', to be the number of ways to rearrange $k$ of $n$ linearly arranged
objects.

{\bf Theorem 2.7.2.} \emph{For $n,k \in \N$, $n^{\underline{k}} = n(n-1)(n-2)\cdots (n - (k-2))(n-(k-1))$.}

\emph{Proof}:  Proceed as in the proof of Theorem 2.7.1, except $i$ ranges from $1$ to $k-1$.  \hfill \rule{2mm}{2mm}

{\bf Theorem 2.9.2.} \emph{For $n, k \in \N$, $\binom{n}{k} = \frac{n!}{k!(n-k)!} = \frac{n^{\underline{k}}}{k!}$.}

\emph{Proof}: Count the number of ways to rearrange $k$ of $n$ objects.  By definition, this is $n^{\underline{k}}$.  But we may perform the
implied task by first selecting which $k$ of the $n$ objects are to be rearranged and then arranging those $k$ objects.
This is done in $\binom{n}{k}k!$ ways.  We have the relationship $n^{\underline{k}} = \binom{n}{k}k!$ which we can solve for the
unknown $\binom{n}{k}$ and obtain $\binom{n}{k} = \frac{n^{\underline{k}}}{k!}$ (the division by $k!$ is possible since $k!$
is never $0$). \hfill \rule{2mm}{2mm}

\item{\bf Combinatorial Arguments.}  The argument above for Theorem 2.9.1 is a \emph{combinatorial argument}.  
Compare it to the following proof which uses Theorem 2.9.2 to introduce an algebraic expression for $\binom{n}{k}$.  
Note that without Theorem 2.9.2, we cannot ostensibly do any algebra with $\binom{n}{k}$.

\emph{Algebraic proof of Pascal's Identity:}  Note that $\binom{n}{k} = \frac{n!}{k!(n-k)!}$, and so 
\begin{eqnarray*} \binom{n-1}{k} + \binom{n-1}{k-1}
&=& \frac{(n-1)!}{k!(n-1-k)!} + \frac{(n-1)!}{(k-1)!(n-1-(k-1))!} \\
&=& \frac{(n-1)!}{k!(n-k-1)!} + \frac{k(n-1)!}{k(k-1)!(n-k)!} \\
&=& \frac{(n-k)(n-1)!}{k!(n-k)(n-k-1)!} + \frac{k(n-1)!}{k!(n-k)!} \\
&=& \frac{(n-k)(n-1)!)}{k!(n-k)!)} + \frac{k(n-1)!}{k!(n-k)!} \\
&=& \frac{(n-k)(n-1)! + k(n-1)!}{k!(n-k)!} \\ 
&= & \frac{n(n-1)! - k(n-1)! + k(n-1)!}{k!(n-k)!} = \frac{n}{k!(n-k)!} = \binom{n}{k}.
\end{eqnarray*}
The first and last equalities are true by Theorem 2.9.2. \hfill $\Box$

A basic distinction a combinatorial argument is that it does not rest solely on algebraic manipulation.  

\item{\bf Multisets.}  A (finite) {\bf multiset} is a function $\mu:S \to \N$, where $S$ is a finite set.
Suppose $S = \{a_1, a_2, \dots, a_k\}$; we call $S$ a {\bf basis set} or {\bf base set} or {\bf ground set}.
It is probably most useful to regard a multiset as a ``\emph{set with repetition of elements allowed}'' even though this is an oxymoron,
and to denote the set as $M = \langle a_1^{\mu(a_1)}, a_2^{\mu(a_2)}, \dots, a_k^{\mu(a_k)}\rangle$,
where $\mu(a_i)$ is the {\bf multiplicity} of element $a_i$. The {\bf size} of a multiset is $\sum_{a_i \in S}^k\mu(a_i)$.
If $M$ is small enough, the exponents can be disregarded and the
set can be explicitly listed.
For example, assume  $S$ includes (at least) the elements $a_1,a_2,$ and $a_3$, $\mu(a_1) = 3$, $\mu(a_3) = 1$, and $\mu(a_i) = 0$ for $i =2$ and $i > 3$.
Then this multiset can be written as $M = \langle a_1,a_1, a_1, a_3\rangle$.


{\bf Fact:} \emph{The number of multisets of size $n$ built from a basis set with $k$ elements is the number of
solutions to $x_1 + x_2 + \cdots + x_k = n$ with $x_i \in \N$.}

\emph{Proof:} Suppose the basis set is $S = \{a_1, \dots, a_k\}$.  Now interpret $x_i$ as $\mu(a_i)$ for each $i$.
Since the size of a multiset is the sum of the multiplicities of the elements from $S$, any vector
$\vec{x} = (x_1, \dots, x_k)$ whose entries sum to $n$ constitutes a specification of a multiset of size $n$. \hfill $\Box$

\item{\bf Multiset Numbers.}  Define $\ds\left(\!\left(k \atop n \right)\!\right)$ to be the number of 
multisets of size $n$ built from a basis set of size $k$.

Given only the definition above you may be able to calculate $\left( \!\left(k \atop n\right)\! \right)$ for limiting values of $k$ and $n$.

Try to argue the following facts (assume $k,n \in \N$, and note that $k < n$ does imply $\left(\!\left(k \atop n \right)\!\right) = 0$):
\begin{enumerate}
\item $\ds\left(\!\left(k \atop 0 \right)\!\right)  = 1$
\item $\ds\left(\!\left(1 \atop n \right)\!\right) = 1$
\item $\ds\left(\!\left(k \atop 1 \right)\!\right) = k$
\item $\ds\left(\!\left(k \atop 2 \right)\!\right) = \binom{k}{2} + k$
\end{enumerate}

Students from a previous semester referred to $\left(\!\left(k \atop n \right)\!\right)$ as
``$n$-choo-choose-$k$'', so I'm going to call them the \emph{choo-choose numbers}.

{\bf Theorem Binochoochoo:} $\left(\!\left(k \atop n \right)\!\right) = \binom{n + k -1}{n}$ for $n, k \in \N$.

\emph{Proof:}  Construct a multiset $M$ of size $n$ from basis set $[k]$ by placing $n$ pennies on a table in a line regarding the pennies as
indistinguishable.  Now take $k-1$ paperclips and place them in between the pennies with any number of paperclips between any pair of
consecutive pennies, to the left of all of the pennies or to the right of all of the pennies.
Read the penny-paperclip code from left to right.  The number of pennies before paperclip 1 is the number of $1$s to be in $M$,
the number of pennies between paperclip 1 and 2 is the number of $2$s in $M$, and so on with the number of pennies to the
right of paperclip k-1 being the number of $k$s in $M$.  You could have created this penny-paperclip code by
marking $n+k-1$ spots on the table and deciding which $n$ spots get the pennies. \hfill $\Box$\\

{\bf Theorem Pascachoochoo:} $\left(\!\left(k \atop n \right)\!\right) = \left(\!\left(k \atop n-1 \right)\!\right)+\left(\!\left(k-1 \atop n \right)\!\right)$.

\emph{Proof:} Assemble a team of $n$ cyclists from $k$ cyclists, one of whom is Lance Armstrong, and pretend you can clone any of the cyclists.
Condition on whether you will include Lance on the team or not.  If you include (at least one copy of) Lance Armstrong, you may
assemble the rest of the team in $\left(\!\left(k \atop n -1 \right)\!\right)$ ways possibly cloning Lance;
If Lance is not to be on the team you can assemble the team in $\left(\!\left(k -1\atop n \right)\!\right)$ ways. \hfill $\Box$\\

\end{enumerate}



\noindent \underline{\hspace{3in}}\\

\noindent{\bf Homework \#4:} ({\bf Due 2/10/2017}).\\ % Dave Rules!

\noindent  Consider functions $f:A \to B$, where $A$ is a finite set with $n$ elements and $B$ is a finite set with $x$ elements.
The table below has 12 entries for each of the possibilities corresponding to various properties $f$, $A$, and $B$ have.
The function $f$ may be injective, surjective, or unrestricted.  The sets $A$ and $B$ may consist of elements that are distinguishable or indistinguishable
(like socks versus shoes).

Recall that, if $A$ and $B$ are sets, the notation $B^A$ stands for the set of all function that map $A$ into $B$.
Recall that $f:A \stackrel{1-1}{\longrightarrow} B$ denotes that $f$ is a function from $A$ into $B$ that is injective.
Recall that $f:A \stackrel{\mbox{onto}}{\longrightarrow} B$ denotes that $f$ is a surjective function mapping $A$ into $B$.

The goal is count the number of functions with the properties indicated by the row and column headings in the table.
For example, in entry number $5$ should be the number of functions that are injective that map $n$ indistinguishable objects to $x$ distinguishable objects.
In Math-porn, this is
$$\left|\left\{f \in B^A : A=\{a_1,a_2, \dots, a_n\}, |B| = x, f:A \stackrel{1-1}{\longrightarrow} B  \right\} \right|.$$
While entry $12$ should house the number
$$\left|\left\{f \in B^A : |A|=n, B =\{b_1,b_2, \dots, b_x\}, f:A \stackrel{\mbox{onto}}{\longrightarrow} B  \right\} \right|.$$

$$\begin{array}{c|c||c|c|c}
\hline A & B & \mbox{unrestricted }& \mbox{injective}& \mbox{onto} \\ \hline \hline
\mbox{distinguishable} & \mbox{distinguishable} & 1. & 2. & 3. \\ \hline
\mbox{indistinguishable}& \mbox{distinguishable}&4.&5.&6. \\ \hline
\mbox{distinguishable} & \mbox{indistinguishable}& 7. & 8. & 9. \\ \hline
\mbox{indistuishable}&\mbox{indistinguishable}& 10. &11.&12. \\ \hline \end{array}$$\\

\noindent Please determine, with proof, the entries 2, 3, 5, 6, and 9 in the table.\\

\noindent \underline{\hspace{3in}}\\


\noindent \underline{\hspace{3in}}\\

\noindent{\bf Homework \#5:} ({\bf Due 2/17/2017}).\\

\begin{enumerate}
\item Recall $\stirling{n}{k}$ denotes the number of partitions of an $n$-set into $k$ blocks.  
Please prove the following  identity (which is a recurrence relation for $\stirling{n}{k}$)
$$\stirling{n}{k} = \stirling{n-1}{k} + k\stirling{n-1}{k},$$
and be sure to indicate the range of $n$ and $k$ and also indicate any relationship between $n$ and $k$ that must hold for the identity to be true.

\item  Suppose $r,s,n \in \N$.  Please prove the identity (\ref{Vandermonde}) in two ways: combinatorially, and algebraically.
\begin{eqnarray} \ds \sum_{i = 0}^n \binom{r}{i}\binom{s}{n-i} = \binom{r+s}{n} \label{Vandermonde} \end{eqnarray}.

\item Suppose $n \in \N$. Please prove $\ds n2^{n-1} = \sum_{i \geq 0}\binom{n}{i}i$ combinatorially and algebraically. 
\end{enumerate}


\noindent \underline{\hspace{3in}}\\

\end{document} 